\documentclass[aspectratio=169,10pt]{beamer}
\usetheme{Madrid}
\usecolortheme{default}

\usepackage[utf8]{inputenc}
\usepackage[T1]{fontenc}
\usepackage{graphicx}
\usepackage{hyperref}
\usepackage{fancyvrb}
\usepackage{xcolor}

\title{VLC for iOS}
\subtitle{Comprehensive Media Player for Apple Platforms}
\author{VideoLAN Team}
\date{\today}

\hypersetup{
    colorlinks=true,
    linkcolor=blue,
    urlcolor=cyan,
}

\begin{document}

\frame{\titlepage}

\begin{frame}
\frametitle{Overview}
\begin{itemize}
    \item Open-source media player for iOS, tvOS, watchOS, and visionOS
    \item Built on VLCKit (libvlc wrapper)
    \item Supports 100+ audio/video formats
    \item Dual-licensed: GPLv2+ and MPLv2
    \item Active development with 1000+ contributors
\end{itemize}
\end{frame}

\begin{frame}
\frametitle{Platform Support}
\begin{columns}
\column{0.5\textwidth}
\textbf{Supported Platforms:}
\begin{itemize}
    \item iOS 12.0+
    \item tvOS 12.0+
    \item watchOS 9.0+
    \item visionOS 1.0+
\end{itemize}

\column{0.5\textwidth}
\textbf{Key Features:}
\begin{itemize}
    \item Multi-format playback
    \item Network streaming
    \item Subtitle support
    \item Audio equalization
    \item Hardware acceleration
\end{itemize}
\end{columns}
\end{frame}

\begin{frame}
\frametitle{Architecture}
\begin{block}{Core Components}
\begin{itemize}
    \item \textbf{VLCKit}: Media playback engine
    \item \textbf{MediaLibraryService}: Media discovery and indexing
    \item \textbf{VLCPlaybackService}: Playback coordination
    \item \textbf{Network Services}: Cloud storage integration
\end{itemize}
\end{block}

\begin{block}{Swift-Objective-C Interop}
\begin{itemize}
    \item Bridging headers for API exposure
    \item Runtime class discovery (\texttt{NSClassFromString})
    \item \texttt{@objc} annotations for visibility
\end{itemize}
\end{block}
\end{frame}

\begin{frame}
\frametitle{Recent Enhancements}
\begin{enumerate}
    \item \textbf{CarPlay Audio}: Smart audio modes with voice boost
    \item \textbf{Focus Mode}: Per-mode library configurations
    \item \textbf{Home Screen Widgets}: Recently played content
    \item \textbf{Live Activities}: Dynamic Island playback controls
    \item \textbf{iCloud Sync}: Cross-device favorites and playlists
\end{enumerate}
\end{frame}

\begin{frame}
\frametitle{CarPlay Audio Enhancements}
\begin{columns}
\column{0.5\textwidth}
\textbf{Audio Modes:}
\begin{itemize}
    \item Normal
    \item Voice (300-3400Hz boost)
    \item Music (bass/treble)
    \item Podcast (voice optimized)
\end{itemize}

\column{0.5\textwidth}
\textbf{Features:}
\begin{itemize}
    \item Real-time equalization
    \item Adjustable voice boost
    \item Frequency-specific processing
    \item Automatic mode switching
\end{itemize}
\end{columns}

\vspace{0.5cm}
\begin{alertblock}{Implementation}
Uses \texttt{VLCCarPlayAudioManager} to interface with \texttt{VLCPlaybackService} for dynamic audio processing.
\end{alertblock}
\end{frame}

\begin{frame}
\frametitle{Focus Mode Integration}
\begin{block}{iOS 15+ Feature}
\begin{itemize}
    \item Per-Focus mode media library filtering
    \item Automatic library configuration changes
    \item Notification-based transitions
    \item Context-aware content presentation
\end{itemize}
\end{block}

\begin{block}{Implementation}
\texttt{VLCFocusModeManager} observes Focus mode changes via \texttt{NSNotificationCenter} and adjusts library presentation accordingly.
\end{block}
\end{frame}

\begin{frame}
\frametitle{Home Screen Widgets}
\begin{columns}
\column{0.5\textwidth}
\textbf{WidgetKit Integration:}
\begin{itemize}
    \item Recently played media display
    \item Configurable update intervals
    \item Multiple widget sizes
    \item SwiftUI-based UI
\end{itemize}

\column{0.5\textwidth}
\textbf{Data Management:}
\begin{itemize}
    \item App Groups for shared storage
    \item Timeline-based updates
    \item Efficient data caching
    \item Background refresh
\end{itemize}
\end{columns}

\vspace{0.5cm}
\begin{exampleblock}{Usage}
Users can add widgets to their Home Screen for quick access to recently played content without opening the app.
\end{exampleblock}
\end{frame}

\begin{frame}
\frametitle{Live Activities}
\begin{block}{iOS 16.1+ Feature}
Dynamic, real-time updates in Dynamic Island and Lock Screen:
\begin{itemize}
    \item Playback progress tracking
    \item Play/pause state
    \item Elapsed and remaining time
    \item Media metadata display
\end{itemize}
\end{block}

\begin{block}{Implementation}
\texttt{VLCLiveActivityManager} uses ActivityKit to manage Live Activity lifecycle, providing persistent controls without app switching.
\end{block}
\end{frame}

\begin{frame}
\frametitle{iCloud Synchronization}
\begin{columns}
\column{0.5\textwidth}
\textbf{Features:}
\begin{itemize}
    \item Favorites sync
    \item Playlist sync
    \item Cross-device sharing
    \item Automatic conflict resolution
\end{itemize}

\column{0.5\textwidth}
\textbf{Technology:}
\begin{itemize}
    \item NSUbiquitousKeyValueStore
    \item CloudKit integration
    \item Background sync
    \item Change notifications
\end{itemize}
\end{columns}

\vspace{0.5cm}
\begin{alertblock}{Benefits}
Users can access their media preferences across all their Apple devices seamlessly.
\end{alertblock}
\end{frame}

\begin{frame}[fragile]
\frametitle{Technical Implementation}
\begin{block}{Swift-Objective-C Interoperability}
\begin{Verbatim}[fontsize=\footnotesize]
// Runtime class discovery
Class widgetClass = NSClassFromString(@"VLCWidgetDataProvider");
id shared = [widgetClass performSelector:@selector(shared)];
[shared performSelector:@selector(updateFromPlaybackService:) 
            withObject:playbackService];
\end{Verbatim}
\end{block}

\begin{block}{Swift Side}
\begin{Verbatim}[fontsize=\footnotesize]
@objc(VLCWidgetDataProvider)
public class VLCWidgetDataProvider: NSObject {
    @objc public static let shared = VLCWidgetDataProvider()
    @objc func updateFromPlaybackService(_ service: Any) { }
}
\end{Verbatim}
\end{block}
\end{frame}

\begin{frame}[fragile]
\frametitle{Code Architecture Example}
\begin{Verbatim}[fontsize=\tiny]
@available(iOS 16.1, *)
@objc(VLCLiveActivityManager)
public class VLCLiveActivityManager: NSObject {
    @objc public static let shared = VLCLiveActivityManager()
    
    @objc(startPlaybackActivityWithTitle:artist:thumbnail:duration:)
    func startPlaybackActivity(title: String, 
                               artist: String?, 
                               thumbnail: String?, 
                               duration: TimeInterval) {
        Task { @MainActor in
            let attributes = VLCPlaybackAttributes(
                title: title, artist: artist ?? ""
            )
            let initialState = VLCPlaybackAttributes.ContentState(
                progress: 0.0, isPlaying: true,
                elapsedTime: 0.0, remainingTime: duration
            )
            let activity = try Activity<VLCPlaybackAttributes>.request(
                attributes: attributes, contentState: initialState
            )
            self.currentActivity = activity
        }
    }
}
\end{Verbatim}
\end{frame}

\begin{frame}
\frametitle{Development Challenges}
\begin{block}{Multi-Target Build}
\begin{itemize}
    \item Shared codebase across platforms
    \item Platform-specific conditionals
    \item Resource bundle management
    \item Dependency configuration
\end{itemize}
\end{block}

\begin{block}{Performance Optimization}
\begin{itemize}
    \item Lazy loading for media discovery
    \item Efficient thumbnail caching
    \item Memory management for large files
    \item Background processing optimization
\end{itemize}
\end{block}
\end{frame}

\begin{frame}
\frametitle{User Experience}
\begin{columns}
\column{0.5\textwidth}
\textbf{Accessibility:}
\begin{itemize}
    \item VoiceOver support
    \item Dynamic Type
    \item High contrast mode
    \item Custom accessibility labels
\end{itemize}

\column{0.5\textwidth}
\textbf{Localization:}
\begin{itemize}
    \item 50+ languages
    \item RTL support
    \item Automatic detection
    \item Cultural adaptations
\end{itemize}
\end{columns}
\end{frame}

\begin{frame}
\frametitle{Open Source Model}
\begin{block}{Licensing}
\begin{itemize}
    \item \textbf{GPLv2+}: GNU General Public License v2 or later
    \item \textbf{MPLv2}: Mozilla Public License v2.0
    \item Dual licensing provides flexibility
    \item Contributors grant relicensing rights
\end{itemize}
\end{block}

\begin{block}{Community}
\begin{itemize}
    \item Public GitLab repository
    \item Active issue tracking
    \item Community contributions
    \item Transparent development
\end{itemize}
\end{block}
\end{frame}

\begin{frame}
\frametitle{Development Workflow}
\begin{block}{Code Quality}
\begin{itemize}
    \item SwiftLint integration
    \item Objective-C style guide
    \item Comprehensive code comments
    \item Consistent naming conventions
\end{itemize}
\end{block}

\begin{block}{Testing}
\begin{itemize}
    \item Manual testing across platforms
    \item GitLab CI integration
    \item Automated screenshot tests
    \item Continuous integration
\end{itemize}
\end{block}
\end{frame}

\begin{frame}
\frametitle{Future Directions}
\begin{itemize}
    \item Enhanced multi-user support for tvOS
    \item Advanced cloud storage integration
    \item Machine learning recommendations
    \item Enhanced CarPlay video support
    \item VisionOS spatial media playback
    \item Advanced audio processing
    \item Enhanced subtitle rendering
\end{itemize}
\end{frame}

\begin{frame}
\frametitle{Key Takeaways}
\begin{block}{Technical Excellence}
\begin{itemize}
    \item Robust architecture with VLCKit
    \item Modern iOS framework integration
    \item Cross-platform compatibility
    \item Performance optimization
\end{itemize}
\end{block}

\begin{block}{User Experience}
\begin{itemize}
    \item Comprehensive feature set
    \item Accessibility-first design
    \item Extensive localization
    \item Modern iOS features
\end{itemize}
\end{block}

\begin{block}{Open Source}
\begin{itemize}
    \item Community-driven development
    \item Transparent process
    \item Dual licensing model
    \item Active contribution
\end{itemize}
\end{block}
\end{frame}

\begin{frame}
\frametitle{Resources}
\begin{block}{Project Links}
\begin{itemize}
    \item \textbf{Repository}: \url{https://code.videolan.org/videolan/vlc-ios}
    \item \textbf{GitHub Mirror}: \url{https://github.com/videolan/vlc-ios}
    \item \textbf{VLCKit}: \url{https://code.videolan.org/videolan/VLCKit}
    \item \textbf{Documentation}: \url{https://www.videolan.org}
\end{itemize}
\end{block}

\begin{block}{Contributing}
\begin{itemize}
    \item Issue tracking on GitLab
    \item Beginner-friendly tags
    \item Code of Conduct
    \item Contribution guidelines
\end{itemize}
\end{block}
\end{frame}

\begin{frame}
\frametitle{Thank You}
\begin{center}
\Large
Questions?

\vspace{1cm}

\large
\textbf{VLC for iOS}\\
Open-source media player for Apple platforms

\vspace{0.5cm}

\normalsize
\url{https://code.videolan.org/videolan/vlc-ios}
\end{center}
\end{frame}

\end{document}

